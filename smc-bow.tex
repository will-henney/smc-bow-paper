\RequirePackage{amsmath}
\documentclass[twocolumn, times]{aastex631}
\usepackage[spanish,es-minimal,english]{babel}
\usepackage[utf8]{inputenc}
\usepackage{natbib}
%\usepackage{microtype}
\usepackage{hyperref}
\usepackage{savesym}
\savesymbol{tablenum}
\usepackage{siunitx}
\restoresymbol{SIX}{tablenum}
\usepackage[varg]{newtxmath}
\usepackage{newtxtext}
\usepackage{booktabs}
\usepackage{array}   % for \newcolumntype macro
\newcolumntype{L}{>{$}l<{$}} % math-mode version of lrc column types
\newcolumntype{R}{>{$}r<{$}} 
\newcolumntype{C}{>{$}c<{$}} 

\bibliographystyle{aasjournal}

\newcommand\ION[2]{#1\,\scalebox{0.9}[0.8]{\uppercase{#2}}}
\newcounter{ionstage}
\renewcommand{\ion}[2]{\setcounter{ionstage}{#2}% 
  \ensuremath{\mathrm{#1\,\scriptstyle\Roman{ionstage}}}}
\newcommand\hii{\ion{H}{2}}
\newcommand\heii{\ion{He}{2}}
\newcommand\hei{\ion{He}{1}}
\newcommand\sii{[\ion{S}{2}]}
\newcommand\oiii{[\ion{O}{3}]}
\newcommand\ariv{[\ion{Ar}{4}]}
\newcommand\Raman{\ensuremath{_{\text{Raman}}}}
\def\th#1#2{\(\theta^{#1}\)\,Ori~#2}
\newcommand\wn{\ensuremath{\tilde{\nu}}}
\newcommand\Wav[1]{\ensuremath{\lambda #1}}

% Chemical formulae
\newcommand*\chem[1]{\ensuremath{\mathrm{#1}}}
% Atomic term symbols
\newcommand\Config[1]{\ensuremath{\mathrm{#1}}}
\newcommand\Term[3]{\ensuremath{\mathrm{#1\ ^{#2}#3}}}
\newcommand\Level[4]{\ensuremath{\mathrm{#1\ ^{#2}#3_{#4}}}}

\newcommand\ha{\ensuremath{\text{H}\alpha}}
\newcommand\hb{\ensuremath{\text{H}\beta}}
\newcommand\lya{\ensuremath{\text{Ly}\alpha}}
\newcommand\lyb{\ensuremath{\text{Ly}\beta}}

\newcommand{\wind}{\ensuremath{_{\text{w}}}}

\begin{document}
\title{A highly ionized stellar bow shock in the Small Magellanic Cloud}
\shorttitle{Highly ionized bow shock in LMC}
\author{William J. Henney, S. Jane Arthur, and M. Valerdi}
\affiliation{%
  \foreignlanguage{spanish}{Instituto de Radioastronomía y
    Astrofísica, Universidad Nacional Autónoma de México, Apartado
    Postal 3-72, 58090 Morelia, Michaoacán, Mexico}
}
\email{w.henney@irya.unam.mx}

\begin{abstract}
  We report the discovery of a parsec-scale stellar bow shock
  associated with the O2\,III(f) star Walborn~3
  in the cluster NGC~346 of the Small Magellanic Cloud.
  Emission line images of \ion{He}{2} and [\ion{Ar}{4}], etc.
\end{abstract}

\keywords{Atomic physics; Circumstellar matter; Stars: winds, outflows}

%\object{M42}

\section{Introduction}
\label{sec:introduction}

The interaction of a star's wind with the surrounding medium
can result in an arc-shaped circumstellar emission nebula,
frequently referred to as a bow shock \citep{Gull:1979a, van-Buren:1988a}.
Stellar bow shocks are found around a wide variety of different stars,
including pre-main sequence stars \citep{Bally:2001a, Henney:2013a},
neutron stars \citep{Cordes:1993a},
and cool giants and supergiants \citep{Sahai:2010a, Cox:2012a},
but they are most commonly associated with hot luminous OB stars
\citep{van-Buren:1995a, Kobulnicky:2016a}.
Bow shocks are most frequently observed via their infrared continuum emission
\citep{Meyer:2016a},
which arises from dust grains that are heated by the
stellar radiation field \citep{Draine:2007a},
but specific classes of bow shock have also been identified
via multiple thermal and non-thermal emission mechanisms
that trace gas and plasma components.
The emission arcs are most commonly interpreted as due to
the hydrodynamic interaction induced by
supersonic relative motion of the star with respect to the ambient material
\citep{Wilkin:1996a},
but models involving a subsonic interaction have also been proposed
\citep{Mackey:2015a, Mackey:2016a}.
Also, the role of the stellar wind ram pressure in supporting the arc
may be replaced by radiation pressure in some cases, see
\citet{Henney:2019a, Henney:2019b, Henney:2019c}. 

Stellar bow shocks can be used to estimate stellar wind mass loss rates
by applying momentum-balance arguments
\citep{Gvaramadze:2012a, Kobulnicky:2018a, Kobulnicky:2019a, Henney:2019c}.
These provide an important check on more traditional spectroscopic methods
\citep{Hillier:2020v},
since the systematic uncertainties and biases are different.
Line-driven wind theory for hot stars
predicts that momentum-loss rates should increase with metallicity, \(Z\),
as \(\dot{M} V\wind \propto Z^{n}\) with \(n = 0.6\)--\(0.8\)
\citep{Vink:2001a, Krticka:2018a, Vink:2021h, Bjorklund:2021k}
for the most luminous stars (\(L > \SI{e6}{L_\odot}\)).

The closest low-metallicity stellar populations
(\(Z = 0.1\)  to \(0.2 Z_\odot\), \citealp{Narloch:2021t})
are found in the Small Magellanic Cloud (SMC) at a distance of
\SI{62}{kpc} \citep{Graczyk:2020g}.
A small number of stellar bow shocks have been
previously identified in the SMC
\citep{Gvaramadze:2011b, Sheets:2013v}
by means of their mid-infrared dust emission.
The majority of these sources are found far from
the cores of dense clusters and are probably \emph{runaways} \citep{Blaauw:1961a},
which have been ejected from a binary system or stellar cluster
\citep{Hoogerwerf:2001a, Renzo:2019b}. 
In the Milky Way, a second class of stellar bow shocks are found
inside young massive star clusters: \emph{weather vanes} \citep{Povich:2008a},
which have low space velocities and are interacting
with streaming motions of the local interstellar medium,
such as champagne flows \citep{Tenorio-Tagle:1979a}.

In this paper, we report the discovery of just such a bow shock
inside the massive stellar cluster NGC~346,
which excites the \hii{} region N66 \citep{Henize:1956v}.

The very early-type star Walborn~3 (W~3) \citep{Walborn:1986y},
also known as MPG~355 \citep{Massey:1989p}
with spectral type ON2 III(f*) \citep{Heydari-Malayeri:2010i}

Atmosphere models of \citet{Rivero-Gonzalez:2012w}


\section{Observations}
\label{sec:observations}

\begin{figure*}
  \centering
  \includegraphics[width=\linewidth]{figs/ngc346-bow-shock-4-panel}
  \caption{
    MUSE emission line images of the cort of NGC~346.
    (a)~High-ionization emission from the bow shock.
    (b)~Medium to low-ionization emission from the surrounding \hii{} region. 
    (c)~Location of the MUSE field within the wider nebula.
    (d)~Zoom on the bow shock region in the light of \ha{} emission,
    }
  \label{fig:muse-acs-multipanel}
\end{figure*}


\begin{figure*}
  \centering
  \includegraphics[width=\linewidth]{figs/ngc346-infrared-multipanel}
  \caption{
    Panoramic view of the NGC~346/N66 region at infrared wavelengths:
    (a)~Short wavelength mid-infrared (\num{3.6} to \SI{8}{\um});
    (b)~Longer wavelength mid-infrared (\num{12} to \SI{100}{\um});
    (c)~Far-infrared (\num{70} to \SI{150}{\um});
    (d)~Zoomed view of panel~c.
    Images are from satellite observatories as follows:
    \textit{Spitzer} IRAC \num{3.6}, \num{4.5}, \SI{8}{\um});
    \textit{WISE} \SI{12}{\um};
    \textit{Spitzer} MIPS \num{24}, \SI{70}{\um};
    \textit{Herschel} PACS \num{100}, \SI{150}{\um}.
    }
  \label{fig:infrared-multipanel}
\end{figure*}

\begin{figure*}[p]
  \centering
  \includegraphics[width=0.9\linewidth]{figs/ngc346-bow-shock-brightness-cuts}
  \caption{
    Emission line surface brightness profiles along an East--West cut across the bow shock,
    derived from MUSE integral field spectra. 
    }
  \label{fig:brightness-cuts}
\end{figure*}


\begin{figure}
  \centering
  \includegraphics[width=\linewidth]{figs/ngc346-bow-shock-he-ratios}
  \includegraphics[width=\linewidth]{figs/ngc346-bow-shock-sii-siii-ne-te}
  \caption{More profiles}
  \label{fig:more-profiles}
\end{figure}

\begin{figure*}[p]
  \centering
  \includegraphics[width=0.9\linewidth]{figs/ngc346-fors1-combo}
  \caption{
    Emission line surface brightness profiles and line ratios along a large-scale
    East--West cut across the entire region, based on FORS1 longslit spectra.
    The slit is close to the symmetry axis of the bow shock.
    (a) Temperature-sensitive line ratio \oiii{} 4363/5007.
    The gray box shows the same inner rim region of the bow shock
    that is highlighted by a gray box in Fig.~\ref{fig:brightness-cuts}.
    (b) Selected emission lines from a wide range of ionization stages. 
    }
  \label{fig:oiii-ratio}
\end{figure*}


\begin{figure}
  \centering
  \includegraphics[width=\linewidth]{figs/ngc346-bow-shock-ariv-diagnostics-annotated}
  \caption{
    Temperature and density diagnostics of the bow shock from \ariv{} line ratios.
    }
  \label{fig:ariv-diagnostics}
\end{figure}

\begin{figure}
  \centering
  \includegraphics[width=\linewidth]{figs/ngc346-bowshock-T-oiii-ariv}
  \includegraphics[width=\linewidth]{figs/ngc346-bowshock-den-ariv}
  \caption{
    Derived temperature of nebula and bow shock. 
    }
  \label{fig:T-oiii-ariv}
\end{figure}

\begin{figure}
  \centering
  \includegraphics[width=\linewidth]{figs/ngc346-infrared-profiles}
  \caption{
    East--west brightness profile cuts in various infrared bands. 
    }
  \label{fig:infrared-profiles}
\end{figure}

\begin{figure}
  \centering
  \includegraphics[width=\linewidth]{figs/ngc346-infrared-sed}
  \caption{
    Spectral energy distribution of bow shock (large symbols and downward arrows)
    and background nebula small symbols joined by dashed line.
    }
  \label{fig:infrared-sed}
\end{figure}



Apart from the inner rim of the bow shock,
there is no diffuse \heii{} emission in the core of NGC~346,
or in the western side of the N66 region.
The eastern side of N66, on the other hand,
shows extensive \heii{} \Wav{4686} emission,
as can be seen at offsets from \num{-200} to \SI{-50}{arcsec}
in Figure~\ref{fig:oiii-ratio}b.
The eastern side of N66 also shows a ten times higher [\ion{Fe}{3}] / \hb{} ratio
and disturbed kinematics in low-ionization lines such as \sii{}.
All these are probably due to a foreground supernova remnant SNR~B0057\(-72.2\) \citep{Ye:1991d}
that overlaps with this part of the nebula \citep{Chu:1988m, Naze:2002q, Danforth:2003m}.




\section{Results}
\label{sec:results}





\section{Conclusions}
\label{sec:conclusions}

\begin{acknowledgments}
  Thank you.
\end{acknowledgments}

\facilities{VLT:Yepun (MUSE)}

\bibliography{smc-bow-refs}

\end{document}

%%% Local Variables:
%%% mode: latex
%%% TeX-master: t
%%% End:
